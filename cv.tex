%-----------------------------------------------------------------------------------------------------------------------------------------------%
%	The MIT License (MIT)
%
%	Copyright (c) 2021 Jitin Nair
%
%	Permission is hereby granted, free of charge, to any person obtaining a copy
%	of this software and associated documentation files (the "Software"), to deal
%	in the Software without restriction, including without limitation the rights
%	to use, copy, modify, merge, publish, distribute, sublicense, and/or sell
%	copies of the Software, and to permit persons to whom the Software is
%	furnished to do so, subject to the following conditions:
%	
%	THE SOFTWARE IS PROVIDED "AS IS", WITHOUT WARRANTY OF ANY KIND, EXPRESS OR
%	IMPLIED, INCLUDING BUT NOT LIMITED TO THE WARRANTIES OF MERCHANTABILITY,
%	FITNESS FOR A PARTICULAR PURPOSE AND NONINFRINGEMENT. IN NO EVENT SHALL THE
%	AUTHORS OR COPYRIGHT HOLDERS BE LIABLE FOR ANY CLAIM, DAMAGES OR OTHER
%	LIABILITY, WHETHER IN AN ACTION OF CONTRACT, TORT OR OTHERWISE, ARISING FROM,
%	OUT OF OR IN CONNECTION WITH THE SOFTWARE OR THE USE OR OTHER DEALINGS IN
%	THE SOFTWARE.
%	
%
%-----------------------------------------------------------------------------------------------------------------------------------------------%

%----------------------------------------------------------------------------------------
%	DOCUMENT DEFINITION
%----------------------------------------------------------------------------------------

% article class because we want to fully customize the page and not use a cv template
\documentclass[a4paper,12pt]{article}

%----------------------------------------------------------------------------------------
%	FONT
%----------------------------------------------------------------------------------------

% % fontspec allows you to use TTF/OTF fonts directly
% \usepackage{fontspec}
% \defaultfontfeatures{Ligatures=TeX}

% % modified for ShareLaTeX use
% \setmainfont[
% SmallCapsFont = Fontin-SmallCaps.otf,
% BoldFont = Fontin-Bold.otf,
% ItalicFont = Fontin-Italic.otf
% ]
% {Fontin.otf}

%----------------------------------------------------------------------------------------
%	PACKAGES
%----------------------------------------------------------------------------------------
\usepackage{url}
\usepackage{parskip} 	

%other packages for formatting
\RequirePackage{color}
\RequirePackage{graphicx}
\usepackage[usenames,dvipsnames]{xcolor}
\usepackage[scale=0.9]{geometry}

%tabularx environment
\usepackage{tabularx}

%for lists within experience section
\usepackage{enumitem}

% centered version of 'X' col. type
\newcolumntype{C}{>{\centering\arraybackslash}X} 

%to prevent spillover of tabular into next pages
\usepackage{supertabular}
\usepackage{tabularx}
\newlength{\fullcollw}
\setlength{\fullcollw}{0.47\textwidth}

%custom \section
\usepackage{titlesec}				
\usepackage{multicol}
\usepackage{multirow}

%CV Sections inspired by: 
%http://stefano.italians.nl/archives/26
\titleformat{\section}{\Large\scshape\raggedright}{}{0em}{}[\titlerule]
\titlespacing{\section}{0pt}{10pt}{10pt}

%for publications
\usepackage[style=authoryear,sorting=ynt, maxbibnames=2]{biblatex}

%Setup hyperref package, and colours for links
\usepackage[unicode, draft=false]{hyperref}
\definecolor{linkcolour}{rgb}{0,0.2,0.6}
\hypersetup{colorlinks,breaklinks,urlcolor=linkcolour,linkcolor=linkcolour}
\addbibresource{citations.bib}
\setlength\bibitemsep{1em}

%for social icons
\usepackage{fontawesome5}

%debug page outer frames
%\usepackage{showframe}

%----------------------------------------------------------------------------------------
%	BEGIN DOCUMENT
%----------------------------------------------------------------------------------------
\begin{document}

% non-numbered pages
\pagestyle{empty} 

%----------------------------------------------------------------------------------------
%	TITLE
%----------------------------------------------------------------------------------------

% \begin{tabularx}{\linewidth}{ @{}X X@{} }
% \huge{Your Name}\vspace{2pt} & \hfill \emoji{incoming-envelope} email@email.com \\
% \raisebox{-0.05\height}\faGithub\ username \ | \
% \raisebox{-0.00\height}\faLinkedin\ username \ | \ \raisebox{-0.05\height}\faGlobe \ mysite.com  & \hfill \emoji{calling} number
% \end{tabularx}

\begin{tabularx}{\linewidth}{@{} C @{}}
\Huge{Mateus da Silva Rosario} \\[7.5pt]
\href{https://github.com/mateusrosario}{\raisebox{-0.05\height}\faGithub\ mateusrosario} \ $|$ \ 
\href{https://linkedin.com/in/mateus-rosario}{\raisebox{-0.05\height}\faLinkedin\ mateus-rosario} \ $|$ \ 
\href{tel:+55063981133108}{\raisebox{-0.05\height}\faMobile \ 063 98113-3108} \ $|$ \
\href{mailto:mateusrosario.me@gmail.com}{\raisebox{-0.05\height}\faEnvelope \ mateusrosario.me@gmail.com} \\
\\
\href{https://mateusrosario.github.io/portfolio/}{\raisebox{-0.05\height} \ \faGlobe \ Meu Portifólio}
\end{tabularx}

%----------------------------------------------------------------------------------------
% EXPERIENCE SECTIONS
%----------------------------------------------------------------------------------------

%Interests/ Keywords/ Summary
\section{Sobre mim}
Sou um \textbf{Cientista da Computação} apaixonado por desenvolvimento e programação. Estou em busca de oportunidades para aprender e crescer profissionalmente. Com 2 anos de experiência em \textbf{desenvolvimento FullStack}, trabalhei na construção de sistemas web e \textbf{aplicativos móveis} em todas as etapas, desde o planejamento até o pós-lançamento. Além disso, aprendi a tomar \textbf{decisões assertivas}, \textbf{organizar o trabalho}, \textbf{colaborar} e \textbf{liderar equipes}. Sou apaixonado por resolver problemas com programação e sempre me empenho em entregar experiências úteis e intuitivas para os usuários.
%----------------------------------------------------------------------------------------
% Experience
%----------------------------------------------------------------------------------------
\section{Experiência}

\begin{tabularx}{\linewidth}{ @{}l r@{} }
\textbf{Estágio - Desenvolvedor FullStack} & \hfill Out. 2020 - Nov. 2022 \\
\textbf{- Fábrica de Software, Universidade Federal do Tocantins}  & \hfill Palmas, TO\\[3.75pt]
\multicolumn{2}{@{}X@{}}{
\begin{minipage}[t]{\linewidth}
    \begin{itemize}[nosep,after=\strut, leftmargin=1em, itemsep=3pt]
        \item[--] Desenvolvimento de Sistema Web para Criação, Elaboração, Revisão e Gerenciamento de PPCs (Projetos Pedagógicos de Curso) na Universidade;
        \item[--] Frameworks Angular, Django e Rest API;
        \item[--] Linguagens JavaScript, TypeScript, Python, HTML, CSS;
        \item[--] Bases de dados MySQL.
    \end{itemize}
    \end{minipage}
}
\end{tabularx}

\begin{tabularx}{\linewidth}{ @{}l r@{} }
\textbf{Estágio Obrigatório - Desenvolvedor FullStack} & \hfill Set. 2019 - Nov. 2019 \\
\textbf{- Universidade Federal do Tocantins}  & \hfill Palmas, TO\\[3.75pt]
\multicolumn{2}{@{}X@{}}{
\begin{minipage}[t]{\linewidth}
    \begin{itemize}[nosep,after=\strut, leftmargin=1em, itemsep=3pt]
        \item[--] Desenvolvimento de Sistema Web para Contabilidade;
        \item[--] Frameworks Angular, Django e Rest API;
        \item[--] Linguagens JavaScript, TypeScript, Python, HTML, CSS;
        \item[--] Bases de dados MySQL.
    \end{itemize}
    \end{minipage}
}
\end{tabularx}

%----------------------------------------------------------------------------------------
% Academic Works
%----------------------------------------------------------------------------------------
\section{Atividades acadêmicas}

\begin{tabularx}{\linewidth}{ @{}l r@{} }
\textbf{Projeto de Extensão Universitária GeoPorTOur} & \hfill Jun. 2020 - Out. 2020 \\
\textbf{- Proex, Universidade Federal do Tocantins}  & \hfill Palmas, TO\\[3.75pt]
\multicolumn{2}{@{}X@{}}{
\begin{minipage}[t]{\linewidth}
    \begin{itemize}[nosep,after=\strut, leftmargin=1em, itemsep=3pt]
        \item[--] Desenvolvimento do aplicativo mobile \href{https://play.google.com/store/apps/details?id=com.fabricadesoftwareuft.geoportour}{GePorTour} para educação social sobre os marcos históricos da cidade de Porto Nacional (Tocantins, Brasil);
        \item[--] Framework Flutter;
        \item[--] Linguagem Dart;
        \item[--] Consumo de API RestFull.
        \item[--] Plataforma Firebase como Banco de Dados;
    \end{itemize}
    \end{minipage}
}
\end{tabularx}

\begin{tabularx}{\linewidth}{ @{}l r@{} }
\textbf{Trabalho publicado em anais de congressos} & \hfill 2018 \\
\textbf{- XI SECCOMP, Universidade Federal do Tocantins}  & \hfill Palmas, TO\\[3.75pt]
\multicolumn{2}{@{}X@{}}{
ROSARIO, M. ; ALMEIDA, M. ; Arruda, W.C. ; LIMA, RAFAEL . Análise comparativa entre AG básico e NSGA-II usando função de Rastrigin. In: XI SECCOMP - Semana Acadêmica do Curso de Ciência da Computação, 2018, Palmas-TO. Anais da XI SECCOMP - Semana Acadêmica do Curso de Ciência da Computação, 2018.
}
\end{tabularx}

\begin{tabularx}{\linewidth}{ @{}l r@{} }
\textbf{Iniciação Científica} & \hfill Jun. 2018 - Jun. 2019 \\
\textbf{- Universidade Federal do Tocantins}  & \hfill Palmas, TO\\[3.75pt]
\multicolumn{2}{@{}X@{}}{
Mateus da Silva Rosario. Avaliação dos algoritmos genéticos transgênico e homogêneo. 2019. Iniciação Científica. (Graduando em Ciência da Computação) - Universidade Federal do Tocantins. Orientador: Rafael Lima de Carvalho.
}
\end{tabularx}


%----------------------------------------------------------------------------------------
%  Habilidades e Competências
%----------------------------------------------------------------------------------------
\section{Habilidades e Competências}
\begin{tabularx}{\linewidth}{@{}l X@{}}
\multicolumn{2}{@{}X@{}}{
\begin{minipage}[t]{\linewidth}
    \begin{itemize}[nosep,after=\strut, leftmargin=1em, itemsep=3pt]
        \item [--] \textbf{Aprendizagem rápida};
        \item [--] Competência para \textbf{trabalho em equipe}, \textbf{liderança} e \textbf{resolução de problemas};
        \item [--] Domínio e conhecimento pleno em \textbf{programação};
        \item [--] Conhecimentos em arquitetura limpa, código limpo e \textbf{domínio de boas práticas};
        \item [--] Experiência com frameworks de \textbf{desenvolvimento ágil (Scrum)};
        \item [--] Domínio nas linguagens de programação \textbf{C, Pyhton, Java, JavaScript, TypeScript, Dart e Swift};
        \item [--] Experiência com \textbf{desenvolvimento mobile} com \textbf{Flutter} e \textbf{SwiftUI}.
        \item [--] Experiência com Banco de dados \textbf{SQL e NoSql} como \textbf{MySQL}, \textbf{Firebase} e \textbf{IBMCloud};
        \item [--] Experiência com Backend em \textbf{Framework Django};
        \item [--] Experiência e Domínio do \textbf{Framework Angular};
        \item [--] Conhecimento e projetos com a biblioteca \textbf{React};
        \item [--] Domínio e experiência com \textbf{HTML e CSS};
        \item [--] Experiência com sistemas \textbf{Frontend e Backend} com comunicação por meio de\textbf{ APIs Resftfull};
        \item [--] Experiência com \textbf{Git} e ferramentas relacionadas;
        \item [--] Conhecimentos em \textbf{redes de computadores}.
        \item [--] Conhecimentos em \textbf{Aprendizagem de Máquina} e \textbf{Inteligência Artificial}.
    \end{itemize}
    \end{minipage}
}
\end{tabularx}

%----------------------------------------------------------------------------------------
%	EDUCATION
%----------------------------------------------------------------------------------------
\section{Educação}
\begin{tabularx}{\linewidth}{lXr}	
2017 - 2023 & Grad. Ciência da Computação pela \textbf{Universidade Federal do Tocantins} & Palmas-TO \\\\

2014 - 2017 & Ensino médio Integrado ao curso Técnico em Agrimensura no \textbf{Instituto Federal do Tocantins} & Palmas-TO \\ 

2010 - 2013 & Ensino Fundamental na \textbf{E. Mun. de Tempo Integral Caroline Campelo Cruz da Silva} & Palmas-TO \\ 
\end{tabularx}


%----------------------------------------------------------------------------------------
%  Languages
%----------------------------------------------------------------------------------------
\section{Idiomas}
\begin{tabularx}{\linewidth}{@{}l X@{}}
\textbf{Inglês Avançado} & \hfill Centro de Idiomas - UFT \\
% \textbf{Alemão Básico} & \hfill  Deutsche Welle Online Course \\  
\end{tabularx}

% %----------------------------------------------------------------------------------------
% %	Plubications
% %----------------------------------------------------------------------------------------
% \section{Atividades Acadêmicas}
% \begin{refsection}[citations.bib]
% \nocite{*}
% \printbibliography[heading=none]
% \end{refsection}

% %----------------------------------------------------------------------------------------
% % Projects
% %----------------------------------------------------------------------------------------
% \section{Projects}

% \begin{tabularx}{\linewidth}{ @{}l r@{} }
% \textbf{Some Project} & \hfill \href{https://some-link.com}{Link to Demo} \\[3.75pt]
% \multicolumn{2}{@{}X@{}}{long long line of blah blah that will wrap when the table fills the column width long long line of blah blah that will wrap when the table fills the column width long long line of blah blah that will wrap when the table fills the column width long long line of blah blah that will wrap when the table fills the column width}  \\
% \end{tabularx}

% %----------------------------------------------------------------------------------------
% %	SKILLS
% %----------------------------------------------------------------------------------------
% \section{Skills}
% \begin{tabularx}{\linewidth}{@{}l X@{}}
% Some Skills &  \normalsize{This, That, Some of this and that etc.}\\
% Some More Skills  &  \normalsize{Also some more of this, Some more that, And some of this and that etc.}\\  
% \end{tabularx}

\vfill
\center{\footnotesize Última Atualização: 13 de Julho de 2023}

\end{document}
